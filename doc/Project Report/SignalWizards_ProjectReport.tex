\documentclass[letterpaper, 11pt, onecolumn, oneside]{article}

%%%%%%%%%%%%%%%%% Preamble %%%%%%%%%%%%%%%%%
% Packages
\usepackage[utf8]{inputenc}
\usepackage{textcomp}
\usepackage{multirow}
\usepackage{graphicx}
\usepackage{amsmath}
\usepackage{fancyhdr}
\usepackage{xcolor}
\usepackage[margin=1in]{geometry}
\usepackage[T1]{fontenc}
\usepackage{enumitem}
% \usepackage[numbered, framed]{matlab-prettifier}
\usepackage{rotating}
\usepackage{hyperref}

% Variables
\newcommand{\thisassignment}{Graduate Project Report}
\newcommand{\myteam}{Signal Wizards}

\renewcommand{\familydefault}{\rmdefault}

% Styling
\definecolor{msu-maroon}{HTML}{660000}

\hypersetup{
    colorlinks=true,
    linkcolor=blue,
    urlcolor=blue,
}

\setlength{\parindent}{0pt}
\setlength{\parskip}{1em}

\fancypagestyle{header}{
    \renewcommand{\headrulewidth}{2pt}
    \renewcommand{\headrule}{\hbox to\headwidth{\color{msu-maroon}\leaders\hrule height \headrulewidth\hfill}}
    \renewcommand{\footrulewidth}{2pt}
    \renewcommand{\footrule}{\hbox to\headwidth{\color{msu-maroon}\leaders\hrule height \footrulewidth\hfill}}
    \fancyhf{}
    \fancyhead[L]{\scriptsize ECE 6413}
    \fancyhead[C]{\LARGE {Digital Signal Processing}}
    \fancyhead[R]{\scriptsize Eric Farmer (EDF63)\\Will Carroll (WOC17)}
    \fancyfoot[L]{\scriptsize DSP - \thisassignment{}}
    \fancyfoot[C]{\scriptsize Page \thepage}
    \fancyfoot[R]{\scriptsize \today}
}
\pagestyle{header}

% Info
\title{
    Virtual Reconstruction of Spacial Reverberation \\
    \Large{Graduate Project Report}
}
\author{
    \begin{tabular}{cc}
        \multicolumn{2}{c}{\textbf{Team Signal Wizards}}                   \\
                                         &                                 \\
        Eric Farmer                      & Will Carroll                    \\
        \href{mailto:edf63@msstate.edu}{\texttt{edf63@msstate.edu}}       & \href{mailto:woc17@msstate.edu}{\texttt{woc17@msstate.edu}}      \\
        Leader                           & Member
    \end{tabular}
}
\date{\today}

% Body
\begin{document}

\maketitle
\newpage

\section*{Abstract}
The objective of this project is to create a system that artificially reconstructs the reverberative properties of an environment.
By capturing the impulse response of the designated space, this system allows the user to simulate the space's acoustics for various inputs.
This ability is useful for sound engineers to test audio performance in different environments without the difficulty of a real-world experiment.
Additionally, this system could reduce configuration time for public address (PA) systems by allowing the user to identify anomalies in the space's frequency response.
This application is beyond the scope of this project, but serves as an example of the potential applications this experiment.
For system performance analysis, the output of the artificial space is compared to the captured output of the real-world space for controlled inputs.

\section*{Introduction}
The objective of this project is to design a system to artificially reconstruct the spacial presence of an environment using the environment's impulse response.
This system has applications in virtual reality \cite{beig2018scalable} where realistic acoustic feedback is necessary for improving the user experience.
Additionally, the system could be applied in live audio performances where the audio engineer must consider the frequency response of the environment.
According to \cite{johnsonperceptually}, proper reverb is crucial to the improved subjective quality of a sound.
Designing a system to artificially reconstruct an environment requires capturing the impulse response and artificially reconstructing the sound.

    \subsection*{Capturing the Impulse Response of a Space}
    The impulse response of the environment is critical to the process of reconstructing its spacial presence.
    Capturing the impulse response is a two-part process --- generate an acoustical impulse and measure the response of the space.
    An acoustical impulse needs to be a nearly instantaneous burst of sound.
    There are specialized devices to generate an acoustical impulse, but these devices are costly and usually impractical for the identified applications of this system.
    Based on research presented by the Audio Engineering Society \cite{abel2010estimating}, popping a balloon can serve as a replacement for acoustical impulse generation devices in most situations.
    Other research suggested a person clapping his hands in the space to create the impulse.
    The balloon pop method was used in this project because of its low cost and ease of repeating performance, which would be difficult if clapping hands.

    \subsection*{Simulating Audio Responses of a Space}
    

\section*{Methodology \& Theory}


\section*{Analysis of Results}


\section*{Conclusion}


\newpage
\bibliography{SignalWizards_ProjectReport}
\bibliographystyle{ieeetr}

\end{document}
